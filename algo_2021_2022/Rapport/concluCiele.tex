	Ce projet d'algorithmique a été pour moi une toute nouvelle expérience. Etant intégrée à l'INSA, je n'avais auparavant jamais fait de projet en informatique (n'en ayant fait en CPGE qu'un seul en mécanique), ni de projet avec un groupe de 4 membres.
	
	Tout d'abord, le projet de correcteur orthographique m'a permis de mettre en application les cours d'algorithmique avancée et de programmation en C dans leur globalité, puisque nous devions réfléchir à la conception du correcteur en suivant les étapes du cycle en V. Cela a été très instructif pour moi, car je n'avais justement jamais appris à suivre ces étapes et je me suis donc rendu compte qu'il était primmordial d'avoir d'abord une idée claire de la structure du code avant de passer au développement ; en particulier lorsqu'il y a autant de fonctions et procédures que dans ce projet.
	
	J'ai également appris à utiliser de nouveaux outils tels que GIT et \LaTeX, ainsi que Doxygen pour la documentation et Valgrind qui nous a été d'une grande aide pour détecter les erreurs de notre code. Cependant, je pense qu'il m'a été difficile d'être efficace au commencement du projet justement à cause de ces différents outils que j'ai mis du temps à maîtriser (ne serait-ce que GIT pour envoyer mes documents aux autres membres du groupe).
	
	Ces deux derniers points (la conception générale du correcteur et la prise en main des nouveaux outils) m'ont donc pris beaucoup de temps. De plus, à cause d'un retard accumulé au cours des séances, nous avons dû consacrer une grande partie de nos vacances sur le projet, ce que je trouve déplorable surtout si l'on prend en compte les contraintes de toutes les autres matières de la ITI3.
	
	Finalement, j'ai trouvé que ce projet était très intéressant d'un point de vue pédagogique mais je regrette de ne pas avoir eu assez de temps pour le finaliser.