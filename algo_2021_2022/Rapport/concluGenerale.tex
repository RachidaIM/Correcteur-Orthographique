En conclusion, nous pouvons dire que ce projet nous aura permis de nous améliorer sur de nombreux points.

\bigskip

Tout d'abord, il nous a permis de nous familiariser avec le C et ses difficultés. En effet, ce langage n'étant pas permissif du tout, nous avons dû réfléchir avec beaucoup de rigueur à chaque aspect du programme, notamment la gestion des pointeurs. Cela a été un des éléments les plus compliqués à aborder et celui qui nous a posé le plus de problèmes au debuggage. Nous avons également découvert les joies de cette étape de debug, qui peut se révéler conséquente pour les programmes de cette envergure.

\bigskip

Autre apport de ce projet, nous avons pu appliquer concrètement des notions d'algorithmiques vues en cours, comme par exemple les arbres ou même les types abstraits de données, que nous n'avions jamais eu l'occasion de créer nous-même. Il était intéressant pour nous d'essayer de mettre en application ces notions de manière pertinente dans l'optique d’optimiser notre programme.

\bigskip

Ensuite, le travail de groupe nous a forcés à adopter une certaine rigueur dans notre approche du développement. Nous avons dû nous répartir les différentes tâches en fonction du niveau de compétence de chaque personne et de sa compréhension du sujet. Le fait de travailler à plusieurs nous a obligés à utiliser GitLab pour orchestrer au mieux le travail de groupe. Concernant l’utilisation de GitLab, cela n’a pas facile au début, mais chacun a réussi à comprendre le logiciel pour l’utiliser au mieux.

\bigskip

Pour conclure, à cause de la charge de travail et la difficulté du projet pour des débutants en C comme nous, nous n’avons pas mené à terme ce premier projet en C.

\bigskip

Nous avons aimé travailler sur ce projet même si malheureusement, nous n’avons pu le finir.