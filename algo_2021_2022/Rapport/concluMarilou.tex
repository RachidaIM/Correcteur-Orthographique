Lors de ce projet, j'ai pu gagner de l'expérience dans différents domaines, et pour cela je l'ai trouvé très intéressant. Le fait de devoir créer de A à Z un projet de cette ampleur représentait un vrai défi pour nous, et j'ai personnellement beaucoup apprécié la phase de conception préliminaire et d'analyse descendante où nous avons été amené à réfléchir par nous même. La spécification des TADs que nous allions utiliser était notamment très intéressante car elle a necessité de vraiment poser le problème et prendre du recul sur le projet.

Le deuxième point sur lequel j'ai pu progresser est la connaissance du langage C. Je n'avais jamais eu l'occasion de coder en C avant, et j'ai pu m'y initier à travers ce projet, ce que je trouve enrichissant pour le futur, étant donné le caractère non permissif du C et la rigueur qu'il exige.

Enfin je trouve que nous avons vraiment pu mesurer ce que signifiait le travail de groupe en informatique. J'avais déjà eu l'occasion de mener un projet à plusieurs en STPI en I3 puis en I4.2 (l'accent avait d'ailleurs été mis sur l'aspect collaboratif pour ce deuxième projet), et pouvoir le refaire au sein du departement ITI est réellement profitable. Nous avons pu constater la rigueur que cela exige dans la conception et dans le développement, et cela nous a permis de nous entraider afin de compenser les faiblesses de chacun.

Pour conclure je dirais que j'ai été contente de réaliser ce projet, le seul problème étant la charge de travail que j'ai trouvé trop conséquente par rapport au cursus ITI3, ce qui ne nous as pas permis de finaliser le programme. 
