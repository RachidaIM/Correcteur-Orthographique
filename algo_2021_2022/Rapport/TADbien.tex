%\documentclass[11pt,a4paper]{article}
%\usepackage{algorithmeUTF8}
%\usepackage{classeRapport}

%\begin{document}


\subsubsection{TAD Mot}
    \begin{tad}
        \tadNom{Mot}
        \tadDependances{\caractere, \chaine, \booleen}
        \begin{tadOperations}{---------------------}
        	\tadOperationAvecPreconditions{creerMot}{\tadUnParam{\chaine}}{\tadUnParam{Mot}}
            \tadOperation{longueur}{\tadUnParam{Mot}}{\tadUnParam{\naturel}}
            \tadOperation{estUnMot}{\tadUnParam{\chaine}}{\tadUnParam{\booleen}}
            \tadOperationAvecPreconditions{iemeCaractere}{\tadDeuxParams{Mot}{\naturelNonNul}}{\tadUnParam{\caractere}}
            \tadOperationAvecPreconditions{remplaceLettre}{\tadTroisParams{Mot}{\caractere}{\naturelNonNul}}{\tadUnParam{Mot}}
			\tadOperationAvecPreconditions{inverseLettres}{\tadDeuxParams{Mot}{\naturelNonNul}}{\tadUnParam{Mot}}
			\tadOperationAvecPreconditions{supprimeLettre}{\tadDeuxParams{Mot}{\naturelNonNul}}{\tadUnParam{Mot}}
			\tadOperationAvecPreconditions{insereLettre}{\tadTroisParams{Mot}{\caractere}{\naturelNonNul}}{\tadUnParam{Mot}}
			\tadOperationAvecPreconditions{decomposeMot}{\tadDeuxParams{Mot}{\naturelNonNul}}{\tadDeuxParams{Mot}{Mot}}
			\tadOperation{estUnCaractereValide}{\tadUnParam{\caractere}}{\tadUnParam{\booleen}}
		\end{tadOperations}

        \begin{tadSemantiques}{---------------------}
            \tadSemantique{creerMot}{Créer un mot à partir d'une chaîne de caractères qui ne contient que des lettres et éventuellement un tiret ou une apostrophe à l'intérieur de la chaîne.}
            \tadSemantique{longueur}{Donne le nombre de caractères contenus dans le mot.}
            \tadSemantique{estUnMot}{Teste si la chaîne de caractère possède bien uniquement des lettres et éventuellement un tiret ou une apostrophe à l'intérieur de la chaîne.}
            \tadSemantique{iemeCaractere}{Renvoie le ième caractère du mot.}
            \tadSemantique{remplaceLettre}{Remplace le ième caractère du mot par un caractère choisi.}
            \tadSemantique{inverseLettres}{Inverse deux lettres consécutives du mot.}
            \tadSemantique{supprimeLettre}{Supprime le ième caractère du mot.}
            \tadSemantique{insereLettre}{Insère un caractère à la ième place du mot.}
            \tadSemantique{decomposeMot}{Décompose le mot en deux mots après le ième caractère.}
            \tadSemantique{estUnCaractereValide}{Teste si un caractère est valide pour être inséré dans un mot.}
        \end{tadSemantiques}

        \begin{tadPreconditions}{Mot}
            \tadPrecondition{creerMot(CdC)}{estUnMot(CdC)}
            \tadPrecondition{iemeCaractere(mot, i)}{i $\leq$ longueur(mot)}
            \tadPrecondition{remplaceLettre(mot, l, i)}{i $\leq$ longueur(mot)}
            \tadPrecondition{inverseLettres(mot, i)}{i $<$ longueur(mot)}
            \tadPrecondition{supprimeLettre(mot, i)}{longueur(mot) $\geq$ 2 et i $\leq$ longueur(mot)}
            \tadPrecondition{insereLettre(mot, l, i)}{i $\leq$ longueur(mot)+1 et estUnCaractereValide(l)}
            \tadPrecondition{decomposeMot(mot, i)}{longueur(mot) $\geq$ 2 et i $\leq$ longueur(mot)-1}
        \end{tadPreconditions}

    \end{tad}


\subsubsection{TAD Dictionnaire}
    \begin{tad}
        \tadNom{Dictionnaire}
        \tadDependances{Mot, \booleen, Fichier}
        \begin{tadOperations}{------------------------}
            \tadOperation{creerDictionnaire}{\tadUnParam{}}{\tadUnParam{Dictionnaire}}
            \tadOperation{estVide}{\tadUnParam{Dictionnaire}}{\tadUnParam{\booleen}}
            \tadOperation{ajouteMot}{\tadDeuxParams{Dictionnaire}{Mot}}{\tadUnParam{Dictionnaire}}
            \tadOperationAvecPreconditions{estPresent}{\tadDeuxParams{Dictionnaire}{Mot}}{\tadUnParam{\booleen}}
            \tadOperationAvecPreconditions{dictionnaireEnFichier}{\tadDeuxParams{Fichier}{Dictionnaire}}{\tadUnParam{Fichier}}
            \tadOperationAvecPreconditions{fichierEnDictionnaire}{\tadDeuxParams{Fichier}{Dictionnaire}}{\tadUnParam{Dictionnaire}}
        \end{tadOperations}

        \begin{tadAxiomes}
            \tadAxiome{estVide(creerDictionnaire)}
            \tadAxiome{ajouteMot(ajouteMot(dico, mot), mot) = ajouteMot(dico, mot)}
            \tadAxiome{estPresent(ajouteMot(dico, mot), mot)}
        \end{tadAxiomes}

        \begin{tadPreconditions}{Dictionnaire}
            \tadPrecondition{estPresent(dico, mot)}{non estVide(dico)}
            \tadPrecondition{dictionnaireEnFichier(fdico, dico)}{estOuvert(fdico)} %Verifier si c'est bien comme ça qu'on fait une précondition, même si c'est une procédure
            \tadPrecondition{fichierEnDictionnaire(fdico, dico)}{estOuvert(fdico)}
        \end{tadPreconditions}
    \end{tad}


\subsubsection{TAD Correcteur Orthographique}
    \begin{tad}
        \tadNom{Correcteur Orthographique}
        \tadDependances{Mot, Dictionnaire}
        \begin{tadOperations}{---------------------------}
            \tadOperation{creerCorrecteur}{\tadUnParam{Dictionnaire}}{\tadUnParam{Correcteur Orthographique}}
            \tadOperationAvecPreconditions{corrigeParInversion}{\tadDeuxParams{Mot}{Correcteur Orthographique}}{\tadUnParam{Correcteur Orthographique}}
            \tadOperation{corrigeParInsertion}{\tadDeuxParams{Mot}{Correcteur Orthographique}}{\tadUnParam{Correcteur Orthographique}}
            \tadOperationAvecPreconditions{corrigeParDecomposition}{\tadDeuxParams{Mot}{Correcteur Orthographique}}{\tadUnParam{Correcteur Orthographique}}
            \tadOperationAvecPreconditions{corrigeParSuppression}{\tadDeuxParams{Mot}{Correcteur Orthographique}}{\tadUnParam{Correcteur Orthographique}} % doit-on mettre une précondition comme quoi le mot ne doit pas avoir qu'une seule lettre?
            \tadOperationAvecPreconditions{corrigeParRemplacement}{\tadDeuxParams{Mot}{Correcteur Orthographique}}{\tadUnParam{Correcteur Orthographique}}
        \end{tadOperations}

        \begin{tadSemantiques}{-------------------------------}
            \tadSemantique{creerCorrecteur}{Crée un correcteur orthorgaphique à partir d'un dictionnaire.}
            \tadSemantique{corrigeParInversion}{Renvoie les corrections possibles du mot si l'on inverse 2 caractères consécutifs.}
            \tadSemantique{corrigeParInsertion}{Renvoie les corrections possibles du mot si on lui insère une lettre.}
            \tadSemantique{corrigeParDecomposition}{Renvoie les corrections possibles du mot si on le décompose en deux mots.}
            \tadSemantique{corrigeParSuppression}{Renvoie les corrections possibles du mot si on lui insère une lettre.}
            \tadSemantique{corrigeParRemplacement}{Renvoie les corrections possibles du mot si on lui insère une lettre.}
        \end{tadSemantiques}

        \begin{tadPreconditions}{Dictionnaire}
            \tadPrecondition{corrigeParInversion(mot, correcteur)}{longueur(mot) $\geq$ 2}
            \tadPrecondition{corrigeParDecomposition(mot, correcteur)}{longueur(mot) $\geq$ 2}
            \tadPrecondition{corrigeParSuppression(mot, correcteur)}{longueur(mot) $\geq$ 2}
            \tadPrecondition{corrigeParRemplacement(mot, correcteur)}{longueur(mot) $\geq$ 2}
        \end{tadPreconditions}
    \end{tad}
    
\subsubsection{Autres TAD utilisés}
	On utilise aussi les 2 types fournis dans le sujet, soit le TAD FichierTexte ainsi que le type Mode.
	


%\end{document}