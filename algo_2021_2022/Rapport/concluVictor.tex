	Pour conclure, j’ai, grâce à ce projet, vu les différents aspects d’un vrai travail de groupe en informatique, car les derniers en date étaient ceux du projet informatique en STPI, mais il n’était pas aussi structuré que celui-ci et j’avais aussi fait un TIPNE au dernier semestre, mais j’étais tout seul.
Étant en équipe, nous nous sommes réparti les tâches et entraidé, ce qui me manquait dans mes autres projets.


	Au niveau informatique, le projet du correcteur orthographique m’a permis de me familiariser avec le langage de programmation C, car je n’en avais jamais fait avant et c’est mon premier projet en C. Ce qui m’a posé le plus de problème avec ce langage, ce sont bien évidemment les pointeurs et aussi de gérer l’espace mémoire lié à ses pointeurs. Nous avons eu beaucoup de « segmentation fault » durant le projet et c’était pour moi les erreurs les plus dures à comprendre.


	En tant que chef de projet, j’ai enfin pu comprendre l’utilisation de Gitlab, je pense qu’il y a encore beaucoup de commandes que je ne connais pas, mais j’ai déjà acquit les bases. Mon groupe était agréable et tout le monde accomplissait ses tâches dans le temps imparti. Nous avons fait régulièrement des réunions, au moins une fois par semaine, pour avoir un suivi de l’évolution du projet et chacun était présent. J’ai donc apprécié travailler avec les autres membres de ce groupe.