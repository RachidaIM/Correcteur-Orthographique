Dans le cadre de notre cours d'Algorithmique Avancée, nous avons dû réaliser un correcteur orthographique codé en langage C. Ce projet a été fait en groupe de 4 et a pour but de nous entraîner non seulement à l'aspect technique du codage en C, mais aussi à tout l'aspect théorique qui concerne toutes les étapes d'un projet informatique. Ainsi nous avons essayé de mener à bien toutes les phases du cycle en V que nous avons vu en cours, depuis l'analyse descendante jusqu'aux derniers tests de validation. Nous avons également été introduit à Git, un nouvel outil qui permet la gestion de code partagé en ligne. Bien que parfois délicate, son utilisation permet de faciliter la coordination entre les membres pendant le développement et la gestion des versions successives du programme.

\bigskip

Dans ce rapport nous détaillons tout d'abord l'analyse du problème qui nous a été posé, puis dans un second temps nous présentons les conceptions préliminaires de nos différents types et fonctions. Ensuite nous exposons les conceptions détaillées des fonctions importantes et de nos types abstraits de données, et enfin nous proposons une implémentation en code C de cet ensemble, ainsi que les tests unitaires correspondants.

\bigskip

\begin{figure}[H]
	\centering 
	\includegraphics[width=\textwidth]{images/matriceTravail.pdf}
	\caption{Matrice qui à fait quoi}
\end{figure}