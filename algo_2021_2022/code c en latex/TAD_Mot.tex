\documentclass{report}
\usepackage[utf8]{inputenc}
\usepackage[T1]{fontenc}
\usepackage[english]{babel}
\usepackage{fullpage}
\usepackage{color}
\usepackage[table]{xcolor}
\usepackage{listings}
 
\definecolor{darkWhite}{rgb}{0.94,0.94,0.94}
 
\lstset{
  aboveskip=3mm,
  belowskip=-2mm,
  backgroundcolor=\color{darkWhite},
  basicstyle=\footnotesize,
  breakatwhitespace=false,
  breaklines=true,
  captionpos=b,
  commentstyle=\color{red},
  deletekeywords={...},
  escapeinside={\%*}{*)},
  extendedchars=true,
  framexleftmargin=16pt,
  framextopmargin=3pt,
  framexbottommargin=6pt,
  frame=tb,
  keepspaces=true,
  keywordstyle=\color{blue},
  language=C,
  literate=
  {²}{{\textsuperscript{2}}}1
  {⁴}{{\textsuperscript{4}}}1
  {⁶}{{\textsuperscript{6}}}1
  {⁸}{{\textsuperscript{8}}}1
  {€}{{\euro{}}}1
  {é}{{\'e}}1
  {è}{{\`{e}}}1
  {ê}{{\^{e}}}1
  {ë}{{\¨{e}}}1
  {É}{{\'{E}}}1
  {Ê}{{\^{E}}}1
  {û}{{\^{u}}}1
  {ù}{{\`{u}}}1
  {â}{{\^{a}}}1
  {à}{{\`{a}}}1
  {á}{{\'{a}}}1
  {ã}{{\~{a}}}1
  {Á}{{\'{A}}}1
  {Â}{{\^{A}}}1
  {Ã}{{\~{A}}}1
  {ç}{{\c{c}}}1
  {Ç}{{\c{C}}}1
  {õ}{{\~{o}}}1
  {ó}{{\'{o}}}1
  {ô}{{\^{o}}}1
  {Õ}{{\~{O}}}1
  {Ó}{{\'{O}}}1
  {Ô}{{\^{O}}}1
  {î}{{\^{i}}}1
  {Î}{{\^{I}}}1
  {í}{{\'{i}}}1
  {Í}{{\~{Í}}}1,
  morekeywords={*,...},
  numbers=left,
  numbersep=10pt,
  numberstyle=\tiny\color{black},
  rulecolor=\color{black},
  showspaces=false,
  showstringspaces=false,
  showtabs=false,
  stepnumber=1,
  stringstyle=\color{gray},
  tabsize=4,
  title=\lstname,
}
 
\begin{document}
 
\begin{lstlisting}
#include <stdlib.h>
#include <assert.h>
#include <string.h>
#include <stdbool.h>
#include "TAD_Mot.h"

char *caracteresValides = "abcdefghijklmnopqrstuvwxyzéèàçù-'";

Mot creerMot(char *CdC){
    Mot mot;
    mot.data = "";
    mot.longueur = 0;
    fixeData(mot, CdC);
    fixeLongueur(mot, strlen(CdC));
    return mot;
}

char *obtenirData(Mot mot){
    return mot.data;
}

void fixeData(Mot mot, char *CdC){
    mot.data=CdC;
}

int longueur(Mot mot){
    return mot.longueur;
}

void fixeLongueur(Mot mot, int l){
    mot.longueur=l;
}

Mot copieMot(Mot mot){
    char *copieData = "";
    Mot copieMot = creerMot("");
    char *data = obtenirData(mot);
    memcpy(copieData, &data, sizeof(char)*(longueur(mot)+1)); // je prends un caractère de plus pour éviter le probleme de la mémoire quand on insère une lettre
    fixeData(copieMot, copieData);
    fixeLongueur(copieMot, longueur(mot));
    return copieMot;
}

bool estUnMot(char *CdC){
    bool mot=true;
    int i=0;
    while (mot && i<strlen(CdC)){
        bool estValide = estUnCaractereValide(CdC[i]);
        if (!(estValide)){
            mot=false;
        }
        i++;
    }
    return mot;
}

char iemeCaractere(Mot mot, int i){
    char *data = obtenirData(mot);
    return data[i];
}

void modifieiemeCaractere(Mot mot, char c, int i){
    mot.data[i]=c; //doit-on mettre des "" ?
}

Mot remplaceLettre(Mot mot, char c, int i){
    Mot testMot=copieMot(mot);
    modifieiemeCaractere(testMot, c, i);
    return testMot;
}

Mot inverseLettres(Mot mot, int i){
    char ci1, ci2;
    Mot testMot=copieMot(mot);
    ci1 = iemeCaractere(testMot, i);
    ci2 = iemeCaractere(testMot, i+1);
    remplaceLettre(testMot, ci2, i);
    remplaceLettre(testMot, ci1, i+1);
    return testMot;
}

Mot supprimeLettre(Mot mot, int i){
    Mot testMot=copieMot(mot);
    char iemeCar = iemeCaractere(testMot, i);
    char *t=&iemeCar;
    memmove(t, t+1, strlen(t+1)+1); // a tester j'ai vu ça sur le net
    fixeLongueur(testMot, longueur(testMot)-1);
    return testMot;
}

Mot insereLettre(Mot mot, char c, int i){
    Mot testMot=copieMot(mot);
    char iemeCar = iemeCaractere(testMot, i);
    char *t=&iemeCar;
    memmove(t+1, t, longueur(testMot)-i);
    modifieiemeCaractere(testMot, c, i);
    fixeLongueur(testMot, longueur(testMot)+1);
    return testMot;
}

Mot *decomposeMot(Mot mot, int i){
    Mot mot1, mot2;
    char *data;
    char data1[i];
    Mot resultat[2];

    data = obtenirData(mot);
    for (int j=0; j<i; j++){
        data1[j]=data[j];
    }
    mot1 = creerMot(data1);
    mot2 = creerMot(data+i);

    resultat[0]=mot1;
    resultat[1]=mot2; //j'essaye de renvoyer un tableau de 2 mots vu qu'en C, on ne peut pas renvoyer 2 choses. je sais pas si ça marche écrit comme ça
    return resultat;
}


bool estUnCaractereValide(char c){
    int i = 1;
    bool estValide = false;
    while ((!estValide) && (i<= strlen(caracteresValides))){
        if (c == caracteresValides[i]){
            estValide = true;
        }
    }
    return estValide;
}
 

\end{lstlisting}
 
\end{document}
